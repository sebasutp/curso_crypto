

\documentclass{beamer}
%\usetheme{mpiis} % Tema de la presentaci�n
\usetheme{Warsaw}
\mode<presentation>
%%%% packages  %%%%
%\usepackage[spanish]{babel}
\usepackage{pgf}
\usepackage[T1]{fontenc}
\usepackage[latin1]{inputenc}
\usepackage{geometry}
\usepackage{shadethm}
\usepackage{amssymb}
\usepackage{amsmath}
\usepackage{amsxtra}
\usepackage{amstext}
\usepackage{latexsym}
\usepackage{dsfont} % Para \mathds{N}
\usepackage[all]{xy}
\usepackage{graphicx}
\usepackage{epsfig}% para los ejemplos con postscript.
\usepackage{verbatim}
\usepackage{hyperref}

\newcommand{\fix}{\marginpar{FIX}}
\newcommand{\new}{\marginpar{NEW}}
\newcommand{\vb}{\mathbf}
\newcommand{\normal}{\mathcal{N}}
\DeclareMathOperator{\Tr}{Tr}
%\renewcommand\bibname{References}

\setlength{\textfloatsep}{5pt}

\begin{document}

  \title[Classic cryptography and cryptanalysis]{Classic cryptography and cryptanalysis}
  \author{Sebastian Gomez-Gonzalez}
  \institute{Max Planck Institute - Intelligent Systems, 
    Germany}
  \frame{\titlepage}

  \section{Introduction}

  \subsection{Motivation}

  \begin{frame}{Motivation}
   \begin{block}{Classic Cryptography}
     \begin{itemize}
       \item Was based more in intuition than mathematical proofs
       \item A good way to introduce cryptography and cryptanalysis
       \item It show us why we should not trust only in intuition for security
       \item Goal: Not clearly defined, but in general is to prevent understanding of
         messages to an adversary.
       \item Inverse: It is also clear, that any cipher should have an inverse to recover
         the original plain-text.
     \end{itemize}
   \end{block}
  \end{frame}

  \section{Substitution ciphers}

  \subsection{Ceasar Cipher}
  \begin{frame}{Ceasar cipher}
    Example: "QEBNRFZHYOLTKCLUGRJMPLSBOQEBIXWVALD"
    \pause
    \begin{block}{How does it work?}
      By shifting the alphabet by some amount (usually 3 characters back)
    \end{block}
  \end{frame}

  \begin{frame}[fragile=singleslide]\frametitle{Ceasar cipher}
   Example: "QEBNRFZHYOLTKCLUGRJMPLSBOQEBIXWVALD"
   \begin{block}{Example}
     \begin{verbatim}
     Plain:    ABCDEFGHIJKLMNOPQRSTUVWXYZ
     Cipher:   XYZABCDEFGHIJKLMNOPQRSTUVW
     \end{verbatim}
   \end{block}
   The decrypted example is: "THE QUICK BROWN FOX JUMPS OVER THE LAZY DOG"
  \end{frame}

  \begin{frame}{Ceasar cipher}
   \begin{block}{Analysis}
     Once you know how it works, it doesn't provide any security!
     \begin{itemize}
       \item Even if you don't know the shift amount, there are only 26 possibilities. Just try them all!
       \item One solution: Use a generic substitution cipher (A permutation of the alphabet)
     \end{itemize}
   \end{block}
  \end{frame}

  \subsection {Substitution cipher}

  \begin{frame}{Subsitution cipher}
    Encrypted text: "SIAAZQLKBAVAZOARFPBLUAOAR"
    \begin{block}{How does it work?}
      Same as Caesar, but instead of taking a shift of the alphabet (26 possible keys) take a permutation
      (That is $26! \approx 2^{88}$ possible keys). Brute force is no longer a good solution! \\
      \pause
      Weakness:
      \begin{itemize}
        \item Same letters in the cipher text correspond to same letters in the plain text
        \item For instance, in the previous text the 'A' appears more often than the other letters (Likely 
          to be an 'E' in the plaintext)
        \item For longer text is easy to break knowing the original language (Bigrams 'TH', Trigrams 'THE', ...)
      \end{itemize}
    \end{block}
  \end{frame}

  \begin{frame}[fragile=singleslide]\frametitle{Subsitution cipher}
    Encrypted text: "SIAAZQLKBAVAZOARFPBLUAOAR"
    \begin{verbatim}
    Plaintext alphabet:  ABCDEFGHIJKLMNOPQRSTUVWXYZ
    Ciphertext alphabet: ZEBRASCDFGHIJKLMNOPQTUVWXY
    \end{verbatim}
    Plain text: "FLEE AT ONCE, WE ARE DISCOVERED"
  \end{frame}

  \section{Vigenere Cipher}

  \subsection{Vigenere Cipher}

  \begin{frame}{Vigenere cipher}
    \begin{block}{How does it work?}
      \begin{itemize}
        \item Give to every letter in the alphabet $\Sigma$ a number between 0 and $|\Sigma|-1$ (In case of English,
          between 0 and 25).
        \item Choose a password for encryption using letters from the alphabet as well
        \item Repeat the password as many times as needed to have the same lenght than the plain text
        \item Add the plain text and the repeated password modulo $|\Sigma|$
      \end{itemize}
    \end{block}
    \begin{tabular}{ c | c  | c  | c  | c  | c  | c  | c  | c  | c  | c  | c  | c}
      a  &  b  &  c  &  d  &  e  &  f  &  g  &  h  &  i  &  j  &  k  &  l  &  m  \\  
      0  &  1  &  2  &  3  &  4  &  5  &  6  &  7  &  8  &  9  &  10  &  11  &  12  \\
      \hline
      n  &  o  &  p  &  q  &  r  &  s  &  t  &  u  &  v  &  w  &  x  &  y  &  z \\
      13 & 14  &  15  &  16  &  17  &  18  &  19  &  20  &  21  &  22  &  23  &  24  &  25
    \end{tabular}
  \end{frame}

  \begin{frame}[fragile=singleslide]\frametitle{Vigenere Example}
    \begin{tabular}{ c | c  | c  | c  | c  | c  | c  | c  | c  | c  | c  | c  | c}
      a  &  b  &  c  &  d  &  e  &  f  &  g  &  h  &  i  &  j  &  k  &  l  &  m  \\  
      0  &  1  &  2  &  3  &  4  &  5  &  6  &  7  &  8  &  9  &  10  &  11  &  12  \\
      \hline
      n  &  o  &  p  &  q  &  r  &  s  &  t  &  u  &  v  &  w  &  x  &  y  &  z \\
      13 & 14  &  15  &  16  &  17  &  18  &  19  &  20  &  21  &  22  &  23  &  24  &  25
    \end{tabular}
    
    \begin{verbatim}
    Key: LEMON
    Plaintext:  ATTACKATDAWN
    Repeated:   LEMONLEMONLE
    Ciphertext: LXFOPVEFRNHR
    \end{verbatim}
  \end{frame}

  \begin{frame}{Vigenere analysis}
    \begin{block}{Strenghts}
      \begin{itemize}
        \item The password can be long enough such that brute force is also impractical
        \item Notice that the same character in the plain text can become different characters in the cipher text
      \end{itemize}
    \end{block}
    \pause
    \begin{block}{Weaknesses}
      \begin{itemize}
        \item If we know part of the plain text, you know part of the key and probably break the rest easily.
        \item But, can we break the Vigenere without knowing part of the text (Like the other methods)?
      \end{itemize}
    \end{block}
  \end{frame}

\end{document}
