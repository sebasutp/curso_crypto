

\documentclass{beamer}
%\usetheme{mpiis} % Tema de la presentaci�n
\usetheme{Warsaw}
\mode<presentation>
%%%% packages  %%%%
%\usepackage[spanish]{babel}
\usepackage{pgf}
\usepackage[T1]{fontenc}
\usepackage[latin1]{inputenc}
\usepackage{geometry}
\usepackage{shadethm}
\usepackage{amssymb}
\usepackage{amsmath}
\usepackage{amsxtra}
\usepackage{amstext}
\usepackage{latexsym}
\usepackage{dsfont} % Para \mathds{N}
\usepackage[all]{xy}
\usepackage{graphicx}
\usepackage{epsfig}% para los ejemplos con postscript.
\usepackage{verbatim}
\usepackage{hyperref}

\newcommand{\fix}{\marginpar{FIX}}
\newcommand{\new}{\marginpar{NEW}}
\newcommand{\vb}{\mathbf}
\newcommand{\normal}{\mathcal{N}}
\DeclareMathOperator{\Tr}{Tr}
%\renewcommand\bibname{References}

\setlength{\textfloatsep}{5pt}

\begin{document}

  \title[Introduction to classic cryptography]{Introduction to classic cryptography}
  \author{Sebastian Gomez-Gonzalez}
  \institute{Max Planck Institute - Intelligent Systems, 
    Germany}
  \frame{\titlepage}

  \section{Introduction}

  \subsection{Motivation}

  \begin{frame}{Privacy}
   \begin{block}{Privacy in the communications}
     \begin{itemize}
       \item Messages can usually be captured.
       \item Internet messages and wireless communications are easy to eavesdrop.
       \item The idea: Encode the message so that only the desired receiver can read it.
     \end{itemize}
   \end{block}
  \end{frame}

  \section{Substitution ciphers}

  \subsection{Ceasar Cipher}
  \begin{frame}{Ceasar cipher}
    Example: "QEBNRFZHYOLTKCLUGRJMPLSBOQEBIXWVALD"
    \pause
    \begin{block}{How does it work?}
      By shifting the alphabet by some amount (usually 3 characters back)
    \end{block}
  \end{frame}

  \begin{frame}[fragile=singleslide]\frametitle{Ceasar cipher}
   Example: "QEBNRFZHYOLTKCLUGRJMPLSBOQEBIXWVALD"
   \begin{block}{Example}
     \begin{verbatim}
     Plain:    ABCDEFGHIJKLMNOPQRSTUVWXYZ
     Cipher:   XYZABCDEFGHIJKLMNOPQRSTUVW
     \end{verbatim}
   \end{block}
   The decrypted example is: "THE QUICK BROWN FOX JUMPS OVER THE LAZY DOG"
  \end{frame}

  \begin{frame}{Ceasar cipher}
   \begin{block}{Analysis}
     Once you know how it works, it doesn't provide any security!
     \begin{itemize}
       \item Encryption = Algorithm + Secret Key
       \item See for example a substitution cipher
     \end{itemize}
   \end{block}
  \end{frame}

  \subsection {Substitution cipher}

  \begin{frame}{Subsitution cipher}
    Encrypted text: "SIAAZQLKBAVAZOARFPBLUAOAR"
    \begin{block}{How does it work?}
      Same as Caesar, but instead of taking a shift of the alphabet (26 possible keys) take a permutation
      (That is $26! \approx 2^{88}$ possible keys). Brute force is no longer a good solution! \\ 
    \end{block}
  \end{frame}

  \begin{frame}[fragile=singleslide]\frametitle{Subsitution cipher}
    Encrypted text: "SIAAZQLKBAVAZOARFPBLUAOAR"
    \begin{verbatim}
    Plaintext alphabet:  ABCDEFGHIJKLMNOPQRSTUVWXYZ
    Ciphertext alphabet: ZEBRASCDFGHIJKLMNOPQTUVWXY
    \end{verbatim}
    Plain text: "FLEE AT ONCE, WE ARE DISCOVERED"
  \end{frame}

  \begin{frame}{Analysis}
    Encrypted text: "SIAAZQLKBAVAZOARFPBLUAOAR" \\
    Plain text: "FLEE AT ONCE, WE ARE DISCOVERED"
    \begin{itemize}
      \item Same letters in the cipher text correspond to same letters in the plain text
      \item For instance, in the previous text the 'A' appears more often than the other letters (Likely 
        to be an 'E' in the plaintext)
      \item For longer text is easy to break knowing the original language (Bigrams 'TH', Trigrams 'THE', ...)
    \end{itemize}
  \end{frame}

  \begin{frame}{Final remarks}
    \begin{block}{Final remarks}
      \begin{itemize}
        \item Design secure encryption algorithms is hard
        \item Use encryption algorithms recommended by experts
      \end{itemize}
    \end{block}    
  \end{frame}

\end{document}
